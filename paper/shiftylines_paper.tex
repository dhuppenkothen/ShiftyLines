% This document is part of the transientdict project.
% Copyright 2013 the authors.

\documentclass[12pt]{emulateapj}
\usepackage{graphicx}
%\usepackage{epsfig}
\usepackage{times}
\usepackage{natbib}
\usepackage{amsfonts}
\usepackage{amsmath}
\usepackage{amsbsy}
\usepackage{bm}
\usepackage{hyperref}
\usepackage{url}
%\usepackage{subfigure}
\usepackage{microtype}
\usepackage{rotating}
\usepackage{booktabs}
\usepackage{threeparttable}
\usepackage{tabularx}
\usepackage{subfigure}


%\usepackage{longtable}%\usepackage[stable]{footmisc}
%\usepackage{color}
%\bibliographystyle{apj}

\newcommand{\project}[1]{\textsl{#1}}
\newcommand{\fermi}{\project{Fermi}}
\newcommand{\rxte}{\project{RXTE}}
\newcommand{\given}{\,|\,}
\newcommand{\dd}{\mathrm{d}}
\newcommand{\counts}{y}
\newcommand{\pars}{\theta}
\newcommand{\mean}{\lambda}
\newcommand{\likelihood}{{\mathcal L}}
\newcommand{\Poisson}{{\mathcal P}}
\newcommand{\Uniform}{{\mathcal U}}
\newcommand{\bg}{\mathrm{bg}}
\newcommand{\word}{\phi}


%\newcommand{\bs}{\boldsymbol}

\begin{document}

\title{Shifty Lines}

\author{D. Huppenkothen\altaffilmark{1, 2}, B. J. Brewer\altaffilmark{3}, Victoria Grinberg\altaffilmark{5}}
 
  \altaffiltext{1}{Center for Data Science, New York University, 726 Broadway, 7th Floor, New York, NY 10003}
  \altaffiltext{2}{E-mail: daniela.huppenkothen@nyu.edu}
  \altaffiltext{3}{Department of Statistics, The University of Auckland, Private Bag 92019, Auckland 1142, New Zealand}
\altaffiltext{4}{MIT}


\begin{abstract}
Stellar winds and winds from accretion disks have Doppler shifts. Sometimes, there can be more than one Doppler shift in 
the same X-ray spectrum. This project solves the problem where (a) we do not know the line amplitudes and widths, (b) the 
number of Doppler shifts in the system and (c) which line belongs to which Doppler shift.

\end{abstract}

\keywords{methods:statistics}

\section{Introduction}

Accretion disks and winds and stuff.

\section{Method}

Each line $i$ is modelled as a Gaussian with location and scale parameters $\mu_i$ and $\sigma_i$, respectively.


\subsection{Assumptions}

\subsubsection{Priors}

Here we list all the priors in the model.


\begin{table*}[hbtp]
\renewcommand{\arraystretch}{1.3}
\footnotesize
\caption{Model Parameters and Prior Probability Distributions}
%\resizebox{\textwidth}{!}{%
\begin{threeparttable} 
\begin{tabularx}{\textwidth}{p{4.0cm}p{7.0cm}X}%lrrrllll}%{lrrrllll}
%\begin{tabular*}{\textwidth}{@{\extracolsep{\fill}} cll}%lrrrllll}%{lrrrllll}
%\begin{tabular}{|l|r|r|l|r|r|r|r|l|}
\toprule
\bf{Parameter} & \bf{Meaning} & \bf{Probability Distribution} \\ \midrule
\it{Hyperparameters} && \\ \midrule
$\mu_\log{A}$ & Mean of Laplacian prior distribution for line amplitude $\log{A}$ &  $\mathrm{Uniform}(-5, 5)$  \\
$\sigma_\log{A}$ & Standard deviation of Laplacian prior distribution for line amplitude $\log{A}$ & $\mathrm{Uniform}(0,2)$ \\
$\mu_\log{q}$ & Mean of Laplacian prior distribution for the $q$-factor $\log{q}$ & $\mathrm{Uniform}(\log(100), \log(1000))$  \\
$\sigma_\log{q}$ & Standard deviation of the Laplacian prior distribution for the $q$-factor $\log{q}$ & $\mathrm{Uniform}(\log(0), \log(0.3))$\\ 
\midrule
\it{Individual Spectral Line Parameters} && \\ \midrule

$\log{A_i}$ & logarithm of the amplitude of line $i$ & $\mathrm{Laplacian}(\mu_A \sigma_A)$ \\
$\log{q_i}$ & logarithm of $q_i = \lambda_\mathrm{line}_i/\sigma_\mathrm{line}_i$ & $\mathrm{Laplacian}(\mu_q, \sigma_q)$ \\
$\log{\counts}_\mathrm{bkg}$ & logarithm of the background flux & $\mathrm{truncated\, Cauchy\, distribution}(10^{-21}, 10^{21})$ \\
$d = v/c$ $ Doppler shift $d$ parametrized as a function of velocity $v$ and speed of light $c$ & $\mathrm{Cauchy\, distribution}(0, 0.001)$
$N$ & Number of possible Doppler shifts & $\mathrm{Uniform}(0,5)$  \\\bottomrule
\end{tabularx}
   \begin{tablenotes}
      %\footnotesize
      \item{An overview over the model parameters and hyperparameters with their respective prior probability distributions.}
     %\item[\emph{a}]{See Section \ref{ch6:priortest} for a discussion on testing an alternative, log-normal prior for spike amplitude and exponential rise time scale.}
     %\item[\emph{a}]{$T_\mathrm{b}$: duration of total burst}
\end{tablenotes}
\end{threeparttable}
\label{tab:priortable}
%\tablecomments{An overview over the model parameters and hyperparameters with their respective prior probability distributions. For parameters where we have explored an alternative distribution in Section 
%\ref{ch6:priortest}, we give parameters and distributions for both priors.}
%\end{sidewaystable}
\end{table*}



\section{Simulated Data}

\subsection{Discussion on where assumptions fail}

Need to talk here about how the likelihood is multi-modal and sometimes the modes can be very narrow. 
This is a problem that makes all sampling algorithms fail, and requires careful tuning.


\section{Observations}

\section{Results}


\section{Discussions}



\paragraph{acknowledgements}
DH was supported by the Moore-Sloan Data Science Environment at NYU. 

\bibliography{td}
\bibliographystyle{apj}

\end{document}


